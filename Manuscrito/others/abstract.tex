\chapter*{Resumen}

Los modelos de difusión son una clase de modelos generativos que han alcanzado el estado del arte en la generación de imágenes. Operan corrompiendo muestras de una distribución desconocida $\ptrue$ hasta otra $\pprior$, de fácil generación, y entrenan una red neuronal para invertir este proceso y recuperar muestras válidas de $\ptrue$.

Sin embargo, esta clase de modelos presenta limitaciones, como la dificultad para transformar $\ptrue$ en distribuciones arbitrarias y trabajar con datos no emparejados. El problema del puente de Schrödinger surge como una alternativa prometedora.

Este problema, basado en transporte óptimo regularizado, busca un proceso estocástico $P$ cercano a un proceso de referencia $R$, cumpliendo con distribuciones marginales predefinidas. Esto permite superar las limitaciones de los modelos de difusión al admitir transformaciones generales y datos no emparejados.

A pesar de su potencial, la literatura existente presenta el puente de Schrödinger de forma compleja. Esta tesis simplifica su comprensión conectándolo con modelos de difusión mediante la teoría de transporte óptimo.

La tesis consta de tres capítulos. El \autoref{dm} analiza los modelos de difusión, sus fundamentos y limitaciones. El \autoref{ot} introduce el transporte óptimo y conceptos como la distancia de Wasserstein. El \autoref{eot_sbp} estudia el transporte óptimo regularizado y su equivalencia con el puente de Schrödinger, facilitando métodos computacionales eficientes.

El apéndice incluye definiciones y resultados técnicos. Aunque desarrollado en $\R^d$ o conjuntos finitos, el marco es extensible a espacios topológicos polacos. Las demostraciones priorizan claridad conceptual, comenzando en contextos discretos y generalizándose al caso continuo.

Los materiales de esta tesis —manuscrito, simulaciones, póster y presentación— están disponibles en el repositorio público de GitHub \href{https://github.com/fernando-fetis/mds-thesis}{\texttt{fernando-fetis/mds-thesis}} para facilitar su reproducibilidad.