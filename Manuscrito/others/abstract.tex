\chapter*{Resumen}

En los últimos años, los modelos de difusión han emergido como una poderosa clase de modelos generativos, alcanzando el estado del arte en diversos dominios, particularmente en la generación de imágenes. Estos modelos operan corrompiendo gradualmente una muestra proveniente de una distribución desconocida $\ptrue$ hasta llegar a otra distribución $\pprior$ de la cual es fácil generar muestras. Para revertir este proceso y generar nuevas muestras desde $\ptrue$, se entrena una red neuronal que aprenda a invertir la dinámica del ruido. Así, una muestra generada desde $\pprior$ puede transformarse en una muestra válida de $\ptrue$ siguiendo el proceso inverso aprendido.

A pesar de su impresionante rendimiento y su capacidad para generar datos de alta calidad, los modelos de difusión presentan limitaciones importantes. Entre estas, destacan la incapacidad de transformar directamente la distribución inicial $\ptrue$ en otra distribución arbitraria y las restricciones asociadas al entrenamiento en escenarios donde los datos no están emparejados. Estas limitaciones han impulsado la búsqueda de alternativas metodológicas, entre las cuales el problema del puente de Schrödinger ha surgido como una solución prometedora.

El problema del puente de Schrödinger, definido en términos del transporte óptimo regularizado, consiste en encontrar un proceso estocástico $P = (P_t)_{t \in [0,1]}$ que, manteniendo cierta proximidad a un proceso de referencia $R = (R_t)_{t \in [0,1]}$ en el sentido de la entropía relativa, cumpla con tener distribuciones marginales predefinidas $\mu$ y $\nu$ en los tiempos $t=0$ y $t=1$, respectivamente. Este enfoque permite superar las limitaciones de los modelos de difusión al admitir transformaciones entre distribuciones arbitrarias y al poder trabajar con datos no emparejados.

Aunque el puente de Schrödinger aborda estas limitaciones de forma efectiva, la literatura existente tiende a presentar el problema desde una perspectiva matemática abstracta, utilizando formalismos técnicos que dificultan su comprensión y adopción en la comunidad de aprendizaje automático. Por ello, el objetivo de esta tesis es cerrar esta brecha mediante un tratamiento exhaustivo y accesible que conecte de manera natural los modelos de difusión con el problema del puente de Schrödinger, utilizando la teoría del transporte óptimo como un puente conceptual y metodológico.

Esta tesis está organizada en tres capítulos principales. En el \autoref{dm}, se realiza una exploración detallada de los modelos de difusión, enfatizando sus fundamentos teóricos, implementaciones prácticas y limitaciones intrínsecas. Este capítulo establece las bases necesarias para comprender los desafíos asociados con estos modelos y justifica la necesidad de enfoques alternativos. 

El \autoref{ot} introduce el problema del transporte óptimo, que busca transformar una distribución de probabilidad en otra minimizando un funcional de costo. Este capítulo presenta conceptos clave como la distancia de Wasserstein, sus propiedades geométricas y su relevancia en la interpolación entre distribuciones. Estas herramientas son esenciales para la comprensión y generalización del problema del puente de Schrödinger.

Finalmente, en el \autoref{eot_sbp}, se estudia una versión regularizada del problema de transporte óptimo, la cual resulta ser equivalente al problema del puente de Schrödinger en su formulación estática. Esta equivalencia permite heredar la teoría y las técnicas del transporte óptimo al análisis y resolución del puente de Schrödinger, facilitando el diseño de métodos computacionalmente eficientes para problemas generativos complejos.

Además, esta tesis incluye un apéndice con definiciones y resultados técnicos utilizados a lo largo del documento. Aunque el marco de trabajo adoptado es $\R^d$ o conjuntos finitos, las definiciones y resultados presentados son fácilmente extensibles a marcos más generales como espacios topológicos polacos. Las demostraciones incluidas tienen un enfoque instructivo, priorizando la claridad conceptual sobre el rigor formal, y muchas de ellas son desarrolladas inicialmente en un contexto discreto para luego generalizarse sin demostración al caso continuo.

Por último, todos los materiales asociados a esta tesis —manuscrito, simulaciones, póster y presentación— están disponibles en el repositorio público de GitHub \href{https://github.com/fernando-fetis/mds-thesis}{\texttt{fernando-fetis/mds-thesis}}, con el objetivo de facilitar la reproducibilidad y la extensión de los resultados aquí presentados.