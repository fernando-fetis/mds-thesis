% Configuración de página:
\geometry{letterpaper,margin=2.5cm,tmargin=3cm}
\fancyhead[L]{}
\fancyhead[R]{\rightmark}
\pagestyle{fancy}
\setlength{\headheight}{13.07225pt}
\addtolength{\topmargin}{-1.07225pt}

% Espaciado entre ecuaciones y texto:
\expandafter\def\expandafter\normalsize\expandafter{%
    \normalsize%
    \setlength\abovedisplayskip{0pt}%
    \setlength\belowdisplayskip{8pt}%
    \setlength\abovedisplayshortskip{-8pt}%
    \setlength\belowdisplayshortskip{2pt}%
}
\renewcommand{\floatpagefraction}{1}%

% Interlineado:
\renewcommand{\baselinestretch}{1.12} 

% Enumeración de ecuaciones y títulos:
\numberwithin{equation}{section}
\setcounter{tocdepth}{1}
\setcounter{secnumdepth}{2}

% Estilo para math environments:
\newtheoremstyle{tesis_env}
{8pt}{8pt}{\upshape}{}{\bfseries}{.}{ }{}

% Estructuras (con shared numbering):
\theoremstyle{tesis_env}
\newtheorem{defn}{Definición}[chapter]
\newtheorem{teo}{Teorema}[chapter]
\newtheorem{prop}{Proposición}[chapter]
\newtheorem{cor}{Corolario}[chapter]

% Nombre de las estructuras (autoref):
\newcommand{\defnautorefname}{Definición}
\newcommand{\teoautorefname}{Teorema}
\newcommand{\propautorefname}{Proposición}
\newcommand{\corautorefname}{Corolario}

% Label y autoref para algorithm:
\makeatletter
\renewcommand{\ALG@name}{Algoritmo}
\makeatother
\newcommand{\algorithmautorefname}{Algoritmo}