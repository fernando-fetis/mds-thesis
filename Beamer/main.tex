\documentclass[aspectratio=43, 10pt]{beamer}

% Idioma y codificación:
\usepackage[spanish]{babel}
\usepackage[utf8]{inputenc}

% Formato de página y diseño:
\usepackage{geometry}
\usepackage{fancyhdr}
\usepackage{parskip}
\usepackage{emptypage}
\usepackage{xcolor}
\usepackage{ragged2e}

% Matemática:
\usepackage{amsmath}
\usepackage{amssymb}
\usepackage{amsthm}
\usepackage{dsfont}

% Figuras:
\usepackage{graphicx}
\usepackage[labelfont=bf]{caption}
\usepackage{tikz}
\usetikzlibrary{arrows.meta, positioning, shapes.multipart}

% Bibliografía y enlaces:
\usepackage[style=apa]{biblatex}
\usepackage{csquotes}
\usepackage{hyperref}

% Algoritmos:
\usepackage{algorithm}
\usepackage{algpseudocode}

\addbibresource{others/references.bib}
\setbeamercovered{transparent}

\usetheme{metropolis}
\metroset{block=fill}
\input{others/shortcuts}
\usepackage{ragged2e}

\apptocmd{\frame}{}{\justifying}{}
\addbibresource{others/references.bib}
\setbeamercovered{transparent}

\usetheme{metropolis}
\metroset{block=fill}

\title{Transporte óptimo y puentes de Schrödinger como generalización de los modelos de difusión}
\author{Fernando Fêtis Riquelme}
\date{Primavera, 2024}
\institute{fcfm - Universidad de Chile}
%\titlegraphic{\hfill\includegraphics[height=1.5cm]{images/uchile}}

\begin{document}

\frame{\titlepage}

\begin{frame}{Tabla de contenidos}
    \setbeamertemplate{section in toc}[sections numbered]
    \tableofcontents[hideallsubsections]
\end{frame}

\section{Modelos de difusión}

\begin{frame}{Modelos de difusión | Proceso forward}
    \uncover<1->{
        El proceso de inyección de ruido es una cadena de Markov en $\R^d$ factorizada de forma causal, cuyos hiperparámetros corresponden a una secuencia finita y decreciente $(\alpha_t)_{t=1}^T\subset(0,1)$:

        \begin{block}{Proceso forward}
            \begin{equation*}
                q(x_{0:T}) = q(x_0)\prod_{t=1}^{T} q(x_t|x_{t-1}),
            \end{equation*}
            con $q(x_0) = \ptrue(x_0)$ y transiciones gaussianas isotrópicas:
            \begin{equation*}
                q(x_t|x_{t-1}) \sim \gaussian{\sqrt{\alpha_t} x_{t-1}}{(1-\alpha_t)\identity{d}}.
            \end{equation*}
        \end{block}
    }
    \uncover<2>{
        La secuencia $(\alpha_t)_{t=1}^T$ debe ser tal que
        \begin{equation*}
            q(x_T)=\int_{\parent{\R^d}^T} q(x_{0:T}) \d x_{0:(T-1)} \approx \pprior(x_T)\sim\gaussian{0}{\identity{d}}.
        \end{equation*}
    }
\end{frame}

\begin{frame}{Modelos de difusión | Proceso backward}
    \uncover<1->{
        El proceso de reconstrucción es otra cadena de Markov factorizada de forma anticausal. Se demuestra que $q(x_{t-1}|x_t,x_0) \sim \gaussian{\mu_q(x_0,x_t,t)}{\sigma_q^2(t)\identity{d}}$, por lo que se propone aprender transiciones gaussianas:
        \begin{block}{Proceso backward}
            \begin{equation*}
                p_\theta(x_{0:T}) = p_\theta(x_T)\prod_{t=1}^{T} p_\theta(x_{t-1}|x_t),
            \end{equation*}
            con $p_\theta(x_T) = \pprior(x_T)$ y transiciones gaussianas:
            \begin{equation*}
                p_\theta(x_{t-1}|x_t)\sim\gaussian{\mu_\theta(x_t,t)}{\Sigma_\theta(x_t,t)}.
            \end{equation*}
        \end{block}
    }
        \begin{itemize}
            \item<2> La función objetivo buscará que $p_\theta(x_{t-1}|x_t) \approx q(x_{t-1}|x_t,x_0)$. Fijando $\Sigma_\theta(x_t,t)=\sigma_q^2(t)\identity{d}$, la función objetivo se reduce a una diferencia de cuadrados.
            \item<3> Con esta elección, solo se necesita aprender el vector de medias de $p_\theta(x_{t-1}|x_t)$ mediante una red neuronal $\mu_\theta:\R^d\times\{0,\ldots,T\}\to\R^d$.
        \end{itemize}
        
\end{frame}

\begin{frame}{Modelos de difusión | Entrenamiento e inferencia}
    \uncover<1>{
        La verosimilitud $p_\theta(x_0)=\int_{\parent{\R^D}^T} p_\theta(x_{0:T}) \d x_{1:T}$ no es computable de forma eficiente. Para entrenar $\mu_\theta$ se maximiza $\E{x_0\sim\ptrue(x_0)}{\elbo(x_0)}$, donde
        \begin{equation*}
            \elbo(x_0) := \log p_\theta(x_0) - \KL{q(x_{1:T}|x_0)}{p_\theta(x_{1:T}|x_0)}.
        \end{equation*}
    }
    \uncover<2>{
        La ELBO se puede evaluar eficientemente:
        \begin{block}{ELBO para DDPM}
            Dada una muestra $x_0\sim\ptrue(x_0)$, entonces:
            \begin{equation*}
                \elbo(x_0) = -\sum_{t=1}^T \frac{1}{2\sigma_q^2(t)} \norm{\mu_q(x_0,x_t, t)-\mu_\theta(x_t,t)}^2 + \cte.
            \end{equation*}
        \end{block}
    }
    \uncover<3->{
        Para la generación de nuevas muestras desde $p_\theta(x_0)$, se simula el proceso backward comenzando con una muestra $x_T\sim\pprior(x_T)$.
    }
\end{frame}

\begin{frame}{Modelos de difusión | Formulación basada en score}
    \uncover<1->{
        La red neuronal $\mu_\theta(x_t,t)$ busca aprender $\mu_q(x_0,x_t,t)$.
    }

    \uncover<2->{
        Se puede demostrar que
        \begin{equation*}
            \mu_q(x_0,x_t,t) = \frac{1}{\sqrt{\alpha_t}}x_t - \frac{1-\alpha_t}{\sqrt{\alpha_t}} \score{x_t}{q(x_t)},
        \end{equation*}
        por lo que $\mu_\theta(x_t,t)$ puede ser reparametrizada por una red neuronal $s_\theta(x_t,t)$ que aprenda directamente la función de score $\score{x_t}{q(x_t)}$.
    }
    \begin{itemize}
        \item<3> Esto conecta los modelos de difusión con SM y con EBM.
        \item<4> Entrega un método de generación condicional (guidance):
              \begin{equation*}
                  \underbrace{\score{x_t}{p_\theta(x_t|y)}}_{\text{score condicional}} = \underbrace{\score{x_t}{p_\theta(y|x_t)}}_{\text{modelo discriminativo}} - \underbrace{\score{x_t}{p_\theta(x_t)}}_{\text{score incondicional}},
              \end{equation*}
              con $p_\theta(y|x_t)$ un clasificador o un modelo tipo CLIP.
    \end{itemize}
\end{frame}

\begin{frame}{Modelos de difusión | Formulación continua}
    La formulación basada en score permite extender los modelos de difusión a tiempo continuo usando SDEs:
    \insertimage{dm/score_sde}{1}{Imagen obtenida desde \cite{song2021scorebased}.}
    La función de costo se puede extender de forma \textit{natural}. También es posible encontrar una expresión análoga a la ELBO en tiempo discreto.
\end{frame}

\begin{frame}{Modelos de difusión | Formulación continua}
    \begin{itemize}
        \item<1> DDPM y DSM son discretizaciones de SDEs específicas.
        \item<2> Se pueden usar diferentes solvers para el proceso backward durante la generación.
        \item<3> Conexión con CNF: existe un proceso determinista con las mismas distribuciones marginales que los procesos de difusión y denoising.
        \only<3>{\insertimage{dm/score_prob_flow}{0.9}{Imagen obtenida desde \cite{song2021scorebased}.}}
        \item<4> En particular, se puede calcular la log-verosimilitud de forma exacta.
    \end{itemize}

\end{frame}

\begin{frame}{Modelos de difusión | Limitaciones}
    \begin{itemize}
        \item<1> Sensibilidad a la elección del proceso de difusión.
        \item<2> No permite transformación entre distribuciones ($\pprior$ fija).
        \item<3> Convergencia asintótica a $\pprior$ y generación lenta.
    \end{itemize}
    \only<1>{
        \begin{figure}
            \includegraphics[width=0.8\textwidth]{images/dm/linear_cosine_scheduler}
            $ $\vspace{0.3cm}
            \includegraphics[width=0.8\textwidth]{images/dm/noise_resolution}
            \caption{Imágenes obtenidas desde \cite{nichol2021improved} y \cite{chen2023importancenoiseschedulingdiffusion}.}
        \end{figure}
    }
    \only<2>{
        \insertimage{dm/cyclegan}{1}{Imagen obtenida desde \cite{zhu2020unpairedimagetoimagetranslationusing}.}
    }
    \only<3>{
        \insertimage{dm/short_time}{1}{Imagen obtenida desde \cite{debortoli2023diffusionschrodingerbridgeapplications}.}
    }
\end{frame}

\section{Problema de Schrödinger}

\begin{frame}{Problema de Schrödinger | Notación}
    \begin{itemize}
        \item<1> Por simplicidad, se asumirá que $\xspace=\yspace\subset\R^d$ es compacto.
        \item<2> Si una medida de probabilidad $\mu\in\probmeasure{\xspace}$ tiene función de densidad $p:\xspace\to\R_+$, entonces
        \begin{equation*}
            \int_\xspace f(x)\d\mu(x)=\int_\xspace f(x)\,p(x)\d x.
        \end{equation*}
    \end{itemize}
\end{frame}

\begin{frame}{Problema de Schrödinger | Formulación dinámica}
    \uncover<1->{
        El siguiente problema consiste en encontrar un proceso estocástico \textit{natural} que transforme una distribución de probabilidad en otra en un horizonte de tiempo finito:

        \begin{block}{SBP (formulación dinámica)}
            El puente de Schrödinger entre dos medidas $\mu,\nu\in\probmeasure{\xspace}$ es el (único) proceso estocástico $P^*$ que resuelve el problema
            \begin{equation*}
                \min_{P\in\Gamma(\mu,\nu)} \KL{P}{W^\epsilon},
            \end{equation*}
            donde $\Gamma(\mu,\nu) := \{P\in \mathcal{C}\parent{[0,1],\xspace}: (P_0\sim\mu) \wedge (P_1\sim\nu)\}$, mientras que $W^\epsilon$ es un movimiento browniano con difusividad $\epsilon$.
        \end{block}
    }
    \begin{itemize}
        \item<2> Notar que se debería definir bien la cantidad $\KL{P}{W^\epsilon}$. No se hará por simplicidad.
        \item<3> La formulación se puede extender a otras medidas de referencias.
    \end{itemize}

\end{frame}

\begin{frame}{Problema de Schrödinger | Formulación dinámica}
    \insertimage{eot_sbp/sbp_solution0.05}{1}{Puente de Schrödinger (dinámico) entre dos distribuciones discretas.}
\end{frame}

\begin{frame}{Problema de Schrödinger | Formulación dinámica}
    \uncover<1,2>{
    Este problema se puede reformular como un problema de control óptimo:
    \begin{equation*}
        \min_{u\in\mathcal{U}} \E{x}{\int_0^1 \frac{1}{2}\norm{u_t(x)}^2 \d t}
        \quad\text{sujeto a}\quad
        \begin{cases}
            \d x_t = u_t(x) \d t + \d W^\epsilon_t\\
            (x_0\sim\mu) \wedge (x_1\sim\nu)
        \end{cases},
    \end{equation*}
    donde $\mathcal{U}$ es el conjunto de controles admisibles.
    }
\begin{itemize}
    \item<2> Se puede reintrepretar el problema como uno de fluidodinámica cambiando la SDE por su ecuación de Fokker-Planck.
    \item<3> La optimalidad también se puede caracterizar por un sistema acoplado de PDEs (sistema de Schrödinger). Esto permite entrenar un modelo neuronal para el SBP mediante máxima verosimilitud.
    \item<4> En este caso, la función objetivo generaliza a la de los modelos de difusión a tiempo continuo, y los modelos de difusión pueden verse como un caso particular del SBP.
\end{itemize}

\end{frame}

\begin{frame}{Problema de Schrödinger | Formulación estática}
    \uncover<1>{
        Se puede demostrar la siguiente descomposición:
        \begin{equation*}
            \KL{P}{W^\epsilon} = \KL{P_{01}}{W^\epsilon_{01}} + \E{(x,y)\sim P_{01}}{\KL{P_{|xy}}{W^\epsilon_{|xy}}}.
        \end{equation*}
    }
    \uncover<2>{
        Luego, el SBP dinámico se puede reducir a un problema estático enfocado únicamente en los extremos del proceso:
        \begin{equation*}
            \underbrace{\min_{P\in\Gamma(\mu,\nu)} \KL{P}{W^\epsilon}}_{\text{problema dinámico}} = \underbrace{\min_{P_{01}\in\Pi(\mu,\nu)} \KL{P_{01}}{W^\epsilon_{01}}}_{\text{problema estático}},
        \end{equation*}
        donde $\Pi(\mu,\nu) := \{\pi\in\probmeasure{\xspace\times\yspace}\,: (\pi_0\sim\mu) \wedge (\pi_1\sim\nu)\}$.
    }

    \uncover<3>{
        Por lo tanto, si $P_{01}^*$ es la (única) solución del SBP estático, la (única) solución del SBP dinámico es
        \begin{equation*}
            P^*(\cdot) = \int_{\xspace\times\yspace} W^\epsilon_{|xy}(\cdot) \d P_{01}^*(x,y).
        \end{equation*}
    }
\end{frame}

\begin{frame}{Problema de Schrödinger | Formulación dinámica}
    \insertimage{eot_sbp/discrete_sinkhorn_graph0.1}{0.7}{Puente de Schrödinger (estático) entre dos distribuciones discretas.}
\end{frame}


\begin{frame}{Problema de Schrödinger | Formulación estática}
    \uncover<1,2>{
        El SBP estático es equivalente al problema de transporte óptimo con regularización entrópica:
    }
    \uncover<2>{
        \begin{align*}
             & \KL{P_{01}}{W_{01}^\epsilon}\\
             & \quad = \frac{1}{\epsilon}\rparent{\underbrace{\int_{\xspace\times\yspace} \frac{1}{2} \norm{x-y}^2 \d P_{01}(x,y)}_{\text{costo de transporte}} + \underbrace{- \epsilon\cdot\entropy{P_{01}}}_{\text{regularizador}}} + \cte,
        \end{align*}

        donde $\entropy{P_{01}}$ es la entropía (diferencial) de $P_{01}$.
    }

    \uncover<3>{
        Más aún, todo SBP (con una cierta medida de referencia) puede ser transformado a un problema de EOT (con una cierta función de costo), y viceversa.
    }

\end{frame}



\section{Transporte óptimo}

\begin{frame}{Transporte óptimo | Problema de Monge}
    \uncover<1,2>{
        El problema de transformar una distribución en otra puede modelarse como un problema de transporte óptimo:
        \begin{center}
            \includegraphics[width=0.3\textwidth]{images/ot/monge_discrete}
            \includegraphics[width=0.69\textwidth]{images/ot/earth_mover}
        \end{center}
    }
    \uncover<2>{
        \begin{block}{Problema de Monge}
            Si $c:\xspace\times\yspace\to\R_+$ es una función continua que mide la \textit{discrepancia} entre dos puntos, el problema de Monge entre $\mu\in\probmeasure{\xspace}$ y $\nu\in\probmeasure{\yspace}$ es:
            \begin{equation*}
                \inf_{\feasible{T:\xspace\to\yspace}{T_\#\mu=\nu}}
                \int_{\xspace} c(x, T(x)) \d\mu(x),
            \end{equation*}
            donde la igualdad $T_\#\mu=\nu$ indica que $T(x)\sim\nu$ cuando $x\sim\mu$.
        \end{block}
    }
\end{frame}

\begin{frame}{Transporte óptimo | Relajación de Kantorovich}
    \uncover<1,2>{
        El problema de Monge es altamente no lineal, no es convexo ni posee necesariamente solución.
    }

    \visible<2>{
        Estas limitaciones se pueden suprimir si se permite división de masa.

        \begin{columns}
            \begin{column}{0.49\textwidth}
                \includegraphics[width=\textwidth]{images/ot/kantorovich_discrete_histogram1}
                \includegraphics[width=\textwidth]{images/ot/kantorovich_discrete_histogram2}
            \end{column}
            \begin{column}{0.5\textwidth}
                % Resuelto usando SciPy.
                \includegraphics[width=\textwidth]{images/ot/kantorovich_discrete_solution}
            \end{column}
        \end{columns}
    }

\end{frame}

\begin{frame}{Transporte óptimo | Relajación de Kantorovich}
    \uncover<1->{
        \begin{block}{Relajación de Kantorovich}  % c convexa?
            Si $c:\xspace\times\yspace\to\R_+$ es una función continua que mide la \textit{discrepancia} entre dos puntos, el problema de Kantorovich entre $\mu\in\probmeasure{\xspace}$ y $\nu\in\probmeasure{\yspace}$ es:
            \begin{equation*}
                \inf_{\pi \in \Pi(\mu,\nu)} \int_{\xspace\times\yspace} c(x, y) \d\pi(x, y),
            \end{equation*}
            donde $\Pi(\mu,\nu) = \{\pi\in\probmeasure{\xspace\times\yspace}\,: \pi_0=\mu, \pi_1=\nu\}$.
        \end{block}
    }
    \begin{itemize}
        \item<2> Es un problema convexo y cumple dualidad fuerte.
        \item<3> Este problema, más un término regularizador, equivale al SBP estático.
    \end{itemize}

\end{frame}

\begin{frame}{Transporte óptimo | Relajación de Kantorovich}
    \begin{columns}
        \begin{column}{0.44\textwidth}
            Bajo hipótesis razonables, ambos problemas son equivalentes en el caso continuo: si $\pi^*\in\probmeasure{\xspace\times\yspace}$ es la solución del problema de Kantorovich, toda su masa está concentrada en una curva, que resulta ser el grafo de la solución $T^*:\xspace\to\yspace$ del problema de Monge.
        \end{column}
        \begin{column}{0.55\textwidth}
            % Usando POT.
            \includegraphics[width=\textwidth]{images/ot/kantorovich_continuous_solution}
        \end{column}
    \end{columns}
\end{frame}

\begin{frame}{Transporte óptimo | Distancia de Wasserstein}
    \uncover<1>{
    Si $d:\xspace\times\xspace\to\R_+$ es una distancia en $\xspace$, entonces
    \begin{equation*}
        \wasserstein{p}{\mu,\nu} = \parent{\inf_{\pi\in\Pi(\mu,\nu)}\int_{\xspace\times\xspace} d(x,y)^p\d\pi}^{\frac{1}{p}}
    \end{equation*}
    es una distancia en $\probmeasure{\xspace}$, llamada \textit{distancia de Wasserstein}.
    }

    \only<2>{
        El espacio métrico $\parent{\probmeasure{\xspace},\mathcal{W}_p}$ es geodésico. Esto permite obtener una formulación dinámica del transporte óptimo. %, obteniendo una dualidad estático-dinámica similar al SBP.

    \insertimage{ot/distr_transform}{1}{Imagen obtenida desde...}
    }
\end{frame}

\begin{frame}{Transporte óptimo | Distancia de Wasserstein}
    \only<1>{
    Esta distancia permite interpolar entre distribuciones de probabilidad a través de geodésicas en $\probmeasure{\xspace}$, generando interpolaciones más realistas que la interpolación euclidiana.
    \begin{center}
        \includegraphics[width=0.8\textwidth]{images/ot/gmm_interpolation}
        \includegraphics[width=0.8\textwidth]{images/ot/gmm_interpolation}
    \end{center}
    }
\end{frame}

\begin{frame}{Transporte óptimo | Distancia de Wasserstein}
    Más aún, se puede realizar interpolación baricéntrica de forma eficiente:
    \begin{center}
        \includegraphics[width=0.9\textwidth]{images/ot/barycenter}
    \end{center}
\end{frame}
    
\begin{frame}{Transporte óptimo | Distancia de Wasserstein}
    La distancia de Wasserstein tiene buenas propiedades matemáticas:
    \begin{itemize}
        \item<1> Es más débil que la distancia en variación total. Más aún, metriza la convergencia débil de medidas.
        \item<2> Puede ser optimizada por redes neuronales: si $g_\theta(z)\sim\mu_\theta$ es un modelo generativo neuronal de variable latente $z\sim\gaussian{0}{\identity{l}}$,  entonces $\theta\mapsto\wasserstein{1}{\mu_\theta,\mu_{\text{true}}}$ es diferenciable (c.t.p.).
        \item<4,5> El problema se puede reformular como uno de control óptimo:
        \begin{equation*}
        \min_{u\in\mathcal{U}} \E{x}{\int_0^1 \frac{1}{2}\norm{u_t(x)}^2 \d t}
        \quad\text{sujeto a}\quad
        \begin{cases}
            \d x_t = u_t(x) \d t\\
            (x_0\sim\mu) \wedge (x_1\sim\nu),
        \end{cases}
    \end{equation*}
    donde $\mathcal{U}$ es el conjunto de controles admisibles.
        \item<5> Formulación de Benamou-Brenier: se puede sustituir la dinámica de $x$ por la ecuación de continuidad, transformando el problema en uno de fluidodinámica.
    \end{itemize}
\end{frame}

\begin{frame}{Transporte óptimo | Modelos de difusión como OT}

    \begin{columns}
        \begin{column}{0.39\textwidth}
            \insertimage{ot/dm_ot}{1}{Imagen obtenida desde }
        \end{column}
        \begin{column}{0.6\textwidth}
            \begin{itemize}
                \item<1> Para un modelo de difusión hasta tiempo $T>0$, se denotará por $E_{T}(x)\in\yspace$ el lugar al que llega un punto $x\sim\ptrue(x)$ que fluye a través de la \textit{probability flow ODE}.
                \item<2> Se ha probado empíricamente que $E_T$ también converge al mapa de Monge entre $\ptrue$ y $\pprior$.
                \item<3> Se han dado contraejemplos teóricos donde esto no ocurre.
            \end{itemize}
        \end{column}
    \end{columns}

\end{frame}

\end{document}